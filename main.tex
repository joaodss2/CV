%%%%%%%%%%%%%%%%%
% This is an sample CV template created using altacv.cls
% (v1.1.4, 27 July 2018) written by LianTze Lim (liantze@gmail.com). Now compiles with pdfLaTeX, XeLaTeX and LuaLaTeX.
% 
%% It may be distributed and/or modified under the
%% conditions of the LaTeX Project Public License, either version 1.3
%% of this license or (at your option) any later version.
%% The latest version of this license is in
%%    http://www.latex-project.org/lppl.txt
%% and version 1.3 or later is part of all distributions of LaTeX
%% version 2003/12/01 or later.
%%%%%%%%%%%%%%%%

%% If you need to pass whatever options to xcolor
\PassOptionsToPackage{dvipsnames}{xcolor}

%% If you are using \orcid or academicons
%% icons, make sure you have the academicons 
%% option here, and compile with XeLaTeX
%% or LuaLaTeX.
% \documentclass[10pt,a4paper,academicons]{altacv}

%% Use the "normalphoto" option if you want a normal photo instead of cropped to a circle
% \documentclass[10pt,a4paper,normalphoto]{altacv}

\documentclass[10pt,a4paper,photo]{altacv}
%% AltaCV uses the fontawesome and academicon fonts
%% and packages. 
%% See texdoc.net/pkg/fontawecome and http://texdoc.net/pkg/academicons for full list of symbols.
%% 
%% Compile with LuaLaTeX for best results. If you
%% want to use XeLaTeX, you may need to install
%% Academicons.ttf in your operating system's font 
%% folder.


% Change the page layout if you need to
\geometry{left=1cm,right=9cm,marginparwidth=6.8cm,marginparsep=1.2cm,top=1.25cm,bottom=1.25cm,footskip=2\baselineskip}

% Change the font if you want to.

% If using pdflatex:
\usepackage[T1]{fontenc}
\usepackage[utf8]{inputenc}
\usepackage[default]{lato}
\usepackage{CJKutf8}
\usepackage{fontawesome}
%\newcommand\faSkype{{\FA\symbol{"F17E}}}
% If using xelatex or lualatex:
% \setmainfont{Lato}

% Change the colours if you want to
\definecolor{Mulberry}{HTML}{72243D}
\definecolor{SlateGrey}{HTML}{2E2E2E}
\definecolor{LightGrey}{HTML}{666666}
\colorlet{heading}{Sepia}
\colorlet{accent}{Mulberry}
\colorlet{emphasis}{Black}
\colorlet{body}{LightGrey}

% Change the bullets for itemize and rating marker
% for \cvskill if you want to
\renewcommand{\itemmarker}{{\small\textbullet}}
\renewcommand{\ratingmarker}{\faCircle}
%% sample.bib contains your publications
\addbibresource{sample.bib}

\usepackage[colorlinks]{hyperref}

%\AtBeginDvi{\input{zhwinfonts}}
\begin{document}

\name{ JOÃO DA SILVA SANTOS }
\tagline{Bsc Software Engineering}

\photo{3.25cm}{joao}
\personalinfo{%
  % Not all of these are required!
  % You can add your own with \printinfo{symbol}{detail}
  
  %\birthday{August 2nd, 1991}
  \email{joaodss@live.com.pt}
  \phone{(+351) 916647105}
  %\mailaddress{Ward No-13, Old Post Office Road,Khajuwala }
  %\location{Braga, Portugal}
  \linkedin{linkedin.com/in/joão-santos-475b1117a}
  
  %% You MUST add the academicons option to \documentclass, then compile with LuaLaTeX or XeLaTeX, if you want to use \orcid or other academicons commands.
%   \orcid{orcid.org/0000-0000-0000-0000}
}

%% Make the header extend all the way to the right, if you want. 
\begin{fullwidth}
\makecvheader
\end{fullwidth}

%% Depending on your tastes, you may want to make fonts of itemize environments slightly smaller
% \AtBeginEnvironment{itemize}{\small}


%% Provide the file name containing the sidebar contents as an optional parameter to \cvsection.
%% You can always just use \marginpar{...} if you do
%% not need to align the top of the contents to any
%% \cvsection title in the "main" bar.

\cvsection[page1sidebar]{Experience}



\cvevent{Critical Techworks, SA - MyAPP}{Backend Software Engineer}{February 2019 -- June 2021}{Porto, Portugal}
\begin{itemize}
\item Contributed as a Backend Engineer in the development of the BMW App using Node.js and C\#.NET, ensuring robust and scalable application architecture.
\item Collaborated within an Agile Scrum team, participated in sprint planning, daily stand-ups, and retrospectives to improve team efficiency and project delivery.
\item Implemented and maintained CI/CD pipelines to automate the deployment process and improve the reliability of microservices, utilizing Terraform and Docker for streamlined infrastructure management.
\item Took the lead in developing and maintaining core microservices, becoming the go-to expert on the team for troubleshooting and optimizing application performance.
\item Engaged in code reviews and mentoring junior developers, fostering a culture of continuous learning and improvement within the team.
\end{itemize}
\divider

\cvevent{Critical Techworks, SA - Road Clearance}{Backend Software Engineer/ DevOps Software Engineer}{June 2021 - October 2024}{Porto, Portugal}
\begin{itemize}
\item Developed a near-real-time data stream service using Scala and Java, enhancing data processing capabilities and improving overall application responsiveness.
\item Maintained and optimized existing Java microservices, focusing on improving metrics and service health status through the construction of new microservices from the ground up,
\item Led the migration of Jenkins pipelines to GitHub Actions, streamlining the CI/CD process and increasing deployment efficiency for the team's microservices.
\item Successfully transitioned all projects from Bitbucket to Github, facilitating improved collaboration and version control within the development team.
\item Managed the migration of microservices from AWS ECS to AWS EKS, optimizing container orchestration and resource management.
\item Oversaw cloud infrastructure management using AWS, leveraging Terraform for infrastructure as code and employing Bash and Python scripts for advanced automation tasks.
\item Reduced Docker image sizes by 80\% optimizing storage and deployment efficiency and implemented a 'Daylight Time Savings' strategy to shutdown Kubernetes pods during off-hours to reduce costs.
\item Developed serverless monitoring solutions using AWS Lambda, and transitioned monitoring tools from CloudWatch to a more robust stack including Grafana, Prometheus and Kibana.
\item Wrote comprehensive documentation for all projects and processes, providing clear guidelines and support for team members on DevOps related topics.

\end{itemize}

\divider

\cvevent{Natixiss - System Team}{DevOps Software Engineer}{October 2024 - Now}{Porto, Portugal}
\begin{itemize}
\item Serve as a DevOps Engineer in a fast-paced team responsible for maintaing and optimizing the infrastructure of key development tools including Jenkins, Bitbucket, Confluence, Jira, Xebia Labs Release, Xebia Labs Deploy, Checkmarkx, ServiceNow, Artifactory, and MyCloud.
\item Act as a central reference point for all DevOps-related inquiries and support within the organization, ensuring seamless operation and integration of development tools.
\item Manage and maintain pre-production and production environments, providing critical support for the infrastructure and applications to need the demands of a demanding development landscape.
\item Collaborate with cross-functional teams to troubleshoot and resolve issues, ensuring minimal downtime and optimal performance of development tools.
\item Implement best practices for CI/CD processes and infrastructure management, while also advocating for modern methodologies within a team accustomed to traditional approaches.
\item Facilitate knowledge sharing and documentation efforts to improve team efficiency and support onboarding processes for new team members.

\end{itemize}

\divider



\vspace{3mm}

\cvsection{Education}

%\cvevent{Cyber Security Academy (357 Hours)}{Rumos}{September 2020 - October 2021}{Lisbon, Portugal}

%\begin{itemize}
%    \item MTA Security Fundamentals
%    \item CompTIA Security+
%    \item Certified Ethical Hacker
%   \item CompTIA Cybersecurity Analyst+
%   \item ISO/IEC 27001
%    \item MoR (Management of Risk) Certification
%    \item Certification Rumos Expert (CRE): Security Auditor
%\end{itemize}
\cvevent{CTeSP in Information Systems}{Politécnico Leiria}{2017-2019}{Leiria, Portugal}
\divider
\cvevent{BSc in Software Engineering}{European Leadership University, Amsterdam}{2021-2024}{Remote, Portugal}
\vspace{7mm}



%%\vspace{7mm}
%%\cvsection{Area of Interest}
%%\begin{itemize}
%%\item Music
%%\item Composition
%%\item Tennis
%%\item Badminton
%%\item Movies
%%\end{itemize}

\medskip

\clearpage

%% If the NEXT page doesn't start with a \cvsection but you'd
%% still like to add a sidebar, then use this command on THIS
%% page to add it. The optional argument lets you pull up the 
%% sidebar a bit so that it looks aligned with the top of the
%% main column.
% \addnextpagesidebar[-1ex]{page3sidebar}
\clearpage
\end{document}
